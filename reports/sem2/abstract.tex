\begin{center}
\thispagestyle{empty}
\vspace*{4\baselineskip}
\LARGE{\textbf{ABSTRACT}}\\[1.0cm]
\end{center}
\thispagestyle{empty}
\large{{The limitation of data representation in today's file systems is that data representation is bound only in a single way of hierarchically organising files. A semantic file system provides addressing and querying based on the content rather than storage location. Semantic tagging is a new way to organise files by using tags in place of directories.  In traditional file systems, symbolic links become non-existent when file paths are changed. Assigning multiple tags to each file ensures that the file is linked to several virtual directories based on its content. By providing semantic access to information, users can organise files in a more intuitive way. In this way, the same file can be accessed through more than one virtual directory. The metadata and linkages for tagging are stored in a relational database which is invisible to the user. This allows efficient searching based on context rather than keywords. The classification of files into various ontologies can be done by the user manually or through automated rules. For certain files types, tags can be suggested by analysing the contents of files. The system would be modular in design to allow customisation while retaining a flexible and stable structure. \\[1cm]}}
Keywords : virtual file system, semantics, indexing, classification, database, tagging, information access, metadata
