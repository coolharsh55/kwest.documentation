\section{APPENDIX F: GLOSSARY}

\noindent \textbf{Acronyms} 
\begin{enumerate}
\item FUSE - File system in Userspace
\item GCC  - GNU Compiler Collection
\item GPL  - General Public License
\item API  - Application programming interface
\item GUI  - Graphical user interface
\item FAQ  - Frequently asked questions 
\item SFS  - Semantic File System
\item VFS  - Virtual File System
\item KFS  - Knowledge File System
\\
\end{enumerate}


\noindent \textbf{Term Definitions} \\
\begin{enumerate}
\item Semantic File System \\
Semantic File Systems are file systems used for information persistence which structure the data according to their semantics and intent, rather than the location as with current file systems. It allows the data to be addressed by their content (associative access) and querying for the data.
\item User Space \\
User space is that portion of system memory in which user processes run. This contrasts with kernel space, which is that portion of memory in which the kernel executes and provides its services.
\item File System \\
A 
file system 
is a 
means to 
organise 
data 
expected to be retained after a program 
terminates by providing procedures to store, retrieve and update data as well as 
manage the available space on the device(s) which contain it. 
\item Virtual File system \\
A virtual file system is an abstraction layer on top of a more concrete file system.
\item Virtual Directory \\
A virtual directory is a directory created in IIS to host our applications and to hide the actual physical location from the application users. 
It may simply designate a folder which appears in a path but which is not actually a sub-folder of the preceding folder in the path. 
\item Meta data \\
Metadata describes how and when and by whom a particular set of data was collected, and how the data is formatted.
\item Symbolic Links and Aliases \\
A symbolic link (also symlink or soft link) is a special type of file that contains a reference to another file or directory in the form of an absolute or relative path. 

%NEW NEW NEW NEW NEW NEW NEW NEW NEW NEW

\end{enumerate}


