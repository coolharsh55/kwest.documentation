\chapter{SOFTWARE IMPLEMENTATION}
\section{Introduction}
The implementation of the software was done in a modular manner using the incremental approach of SDLC. ANSI C wasa the language used to implement the project. An SQLite database was used for creating and managing the data repository. Various external libraries were used for the extraction of the metadata. The following modules consitituted our project:
\begin{list}{•}{•}
\item \textbf{Module 1:} Creation of a virtual file system using FUSE
\item \textbf{Module 2:} Interfacing a Data Repository using SQLite
\item \textbf{Module 3:} Adding automated Extraction of Metadata 
\item \textbf{Module 4:} Importing Semantics in to the file system
\item \textbf{Module 5:} Exporting Semantics from the file system
\item \textbf{Module 6:} Association Rule learning using Apriori Algorithm
\end{list}

\section{Creation of a virtual file system using FUSE}

The first module of implementation was to create a basic file system using FUSE. Using FUSE we created a virtual file system capable of doing all the operations that a normal file system does.
The module consists of the following phases:
\begin{list}{•}{•}

\item \textbf{Phase 1:} Implement FUSE to create a basic file system structure using ANSI C as the implementation language.

\item \textbf{Phase 2:} Connect the file system created to the data repository created using a SQLite database. Create tags and files view by quering the database through FUSE.

\item \textbf{Phase 3:} Implementing common file operations with respect to tags and files such as read, write, open, copy, move etc.

\item \textbf{Phase 4:} Extraction of metadata from the files using external libraries and organisation of the file system based on metadata.

\item \textbf{Phase 5:} Displaying suggestion based on associations derived using Apriori Algorithm. 
\end{list}

\section{Interfacing a Data Repository using SQLite}
The project was to implement a semantic file system. This required a data repository to store all the information such as file name, physical location, attributes, etc. Also, the metadata extracted from the files was also stored in the database. Thus, the database formed a central information location for the file system.


It is vital for the proper functioning of the system that the database always remains consistent. Logging mechanisms ensure that operations on the database always reach an endpoint. This module is used to check, correct and maintain integrity of the database by checking for redundant entries. Also, if there are new files which have not been added to KWEST, this module can help the user add them.


The database is an important module of the file system. All the data required to browse and navigate the filesystem is stored in the database. FUSE interacts with the data in the database by quering for particular data based on path accessed. We implement this module in the following ways:
\begin{list}{•}{•}
\item \textbf{Phase 1:} Create database tables for a file system.
\item \textbf{Phase 2:} The relation tables between tags, files are stored.
\item \textbf{Phase 3:} Store the extracted metadata in the database.
\item \textbf{Phase 4:} The association rules for the data are derived using the Apriori algorithm.
\end{list}

\section{Adding automated Extraction of Metadata}
Metadata (meta content) is defined as data providing information about one or more aspects of the data. Metadata can be stored either internally, in the same file as the data, or externally, in a separate file. Metadata that is embedded with content is called embedded metadata.
The metadata of the file is extracted by using external libraries. The data repository stores the extracted metadata in a predetermined format. 
\begin{list}{•}{•}
\item \textbf{Phase 1:} Test external libraries to determine which of them can be used.
\item \textbf{Phase 2:} Extract metadata using external libraries.
\item \textbf{Phase 3:} Store the extracted metadata in the database.
\item \textbf{Phase 4:} Form relations between metadata and files using association rules.
\end{list}

\section{Importing Semantics in to the file system}
Users already have certain organizational structures in the way they store data in file systems. This module imports semantics by converting the storage hierarchy to tag-based hierarchy. This means the directory structure present in the file system will be used to form tags and the files listed under the directory are tagged under that tag.
\begin{list}{•}{•}
\item \textbf{Phase 1:} Parse the folder structure on local hard disk.
\item \textbf{Phase 2:} Add entry for each file and folder to the database.
\item \textbf{Phase 3:} Remove or ignore hidden and system files.
\item \textbf{Phase 4:} Prune the database entries on every start up.
\end{list}

\section{Exporting Semantics from the file system}
This module can export the storage hierarchy to some external location. The semantic organization of tags is converted to actual directories and the files are then copied to these directories. This is similar to copying contents from one file system to another.
\begin{list}{•}{•}
\item \textbf{Phase 1:} Copy virtual locations to external location.
\item \textbf{Phase 2:} Perform physical copy of files.
\item \textbf{Phase 3:} Create folders and subfolders based on tags.
\item \textbf{Phase 4:} Copy suggestions using data repository.
\end{list}

\section{Association Rule learning using Apriori Algorithm}
Association rules help in organising the file system data by providing suggestions while tagging files. These suggestions can be helpful when the user has either forgotten to tag the file, or is yet about to do it. This association rule learning approach uses the Apriori algorithm.
\begin{list}{•}{•}
\item \textbf{Phase 1:} Run Apriori over the KWESt database.
\item \textbf{Phase 2:} Perform optimisations and prune steps.
\item \textbf{Phase 3:} Store association rules in database.
\item \textbf{Phase 4:} Integrate with KWESt to show suggestions.
\end{list}

\section{User Interface}
A specialised user interface could have provided advantages like tag view, KWEST operation support, etc. But, it increases the requirements and limits usage to that software only. Therefore, KWEST was designed to be usable on any regular file manager. Also, the terminal forms an important part of the Linux users arsenal. The KWEST operating system can also be easily used through the terminal. The file system is designed to be used as any traditional file system. 
