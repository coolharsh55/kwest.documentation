\chapter {SOFTWARE TESTING}
\section{Introduction}
\subsection{Purpose}
Software testing can be stated as the process of validating and verifying that a computer program/application/product:
\begin{itemize}
\item Meets the requirements that guided its design and development.
\item Works as expected.
\item Can be implemented with the same characteristics.
\item Satisfies the needs of stakeholders.
\end{itemize}

Software testing, depending on the testing method employed, can be implemented at any time in the development process. Traditionally most of the test effort occurs after the requirements have been defined and the coding process has been completed, but in the Agile approaches most of the test effort is on-going. As such, the methodology of the test is governed by the chosen software development methodology.

\subsection{Scope}
The testing of the system was done manually and no testing tools were used. The \emph{test plan} describes the \emph{unit, functional, performance, usability, regression} tests that were performed. Only codes that were pushed as commits were considered as candidates for testing.

\subsection{Intended Audience}
The testing of this system is intended for 3 types of audiences:
\begin{enumerate}
\item \textbf{End Users:} The users who will be using the system will review the testing as a mark of stability and performance of the system.
\item \textbf{Developers:} Will view the testing for knowing existing limitations and bugs.
\item \textbf{Reviewers:} Will use these test results as a metric to evaluate the project.
\end{enumerate}

\section{Test Plan}
\subsection{Target Items}
The following have been identified as targets for testing:
\begin{enumerate}
\item Code and associated areas
\item Filesystem operations
\item Databases: SQLite3
\item Operating Systems
\end{enumerate}

\subsection{Outline of Tests}
\subsection*{Tests performed}
\begin{enumerate}
\item Performance tests
\item Functional tests
\item Data Integrity tests
\item Regression tests
\item Usability tests
\end{enumerate}

\subsection{Test Approach}
Any bugs found should be reported with related information, which should include
who discovered it, how, a description of the bug, and who fixed it and
when. Also, re-testing of the code done to make sure that defect has been fixed
and there no new bugs produced due to change in code.

\begin{enumerate}
\item \textbf{Performance Testing:} \\
The focus of Performance testing is checking a software program’s
\begin{itemize}
\item \emph{Speed} : Determines whether the system responds quickly.
\item \emph{Scalability} : Determines maximum user load the software application can handle.
\item \emph{Stability} : Determines if the application is stable under varying loads.
\end{itemize}
\textbf{Tools required:} Software timers \\
\textbf{Success criteria:} 
\begin{enumerate}
\item Manual(user) perception does not notice any ``lags''.
\item Time to perform operations is within an acceptable range.
\end{enumerate}


\item \textbf{Functional Testing:} \\
The prime objective of Functional testing is   checking the functionalities of the software system. It mainly concentrates on -
\begin{itemize}
\item \emph{Mainline functions} :  Testing the main functions of an application.
\item \emph{Basic Usability} : It involves basic usability testing of the system. It checks whether an user can freely navigate through the screens without any difficulties.
\item \emph{Accessibility} :  Checks the accessibility of the system for the user.
\item \emph{Error Conditions} : Usage of testing techniques to check for error conditions.  It checks whether suitable error messages are displayed.
\end{itemize}
\textbf{Tools required:} None(manual testing) \\
\textbf{Success criteria:} All of the following are successfully tested:
\begin{enumerate}
\item all key use-case scenarios.
\item all key features.
\end{enumerate}

\item \textbf{Data Integrity Testing:} \\
Data integrity refers to the quality of the data in databases and is the measurement by which users examine data quality, reliability and usefulness. Data integrity testing verifies that converted data is accurate and functions correctly within a given application. Testing data integrity involves:
\begin{itemize}
\item \textbf{Database} : Verifying that correct values are saved in databases.
\item \textbf{Write-back} : Correct data is written to disk.
\item \textbf{Read} : Correct data is read from disk.
\item \textbf{File Integrity} : Operations do not break existing files.
\end{itemize}
\textbf{Tools required:} File compare tools (manual testing) \\
\textbf{Success criteria:} All of the following are successfully tested:
\begin{enumerate}
\item files are same in size, byte-blocks, permissions and parameters.
\item data is not changed, modified or removed unless intended.
\end{enumerate}

\item \textbf{Regression Testing:} \\
Regression Testing is required when there is a :
\begin{itemize}
\item Change in requirements and code is modified according to the requirement
\item New feature is added to the software
\item Defect fixing
\item Performance issue fix 
\end{itemize}
\textbf{Tools required:} None(manual testing) \\
\textbf{Success criteria:} All of the following are successfully tested:
\begin{enumerate}
\item all previous operations are successfully executed.
\item previously sovled bugs are not re-introduced.
\item operations do not suffer from unwanted performance hits.
\end{enumerate}

\item \textbf{Usability Testing:} \\
Goal of this testing is to satisfy users and it mainly concentrates on the following parameters of a system: \\
\textbf{Effectiveness of the system}
\begin{itemize}
\item Is the system is easy to learn?
\item Is the system useful and adds value to the target audience?
\item Is Content, Color, Icons, Images used are aesthetically pleasing ?
\end{itemize}
\textbf{Efficiency}
\begin{itemize}
\item Navigation required to reach desired screen/webpage should be very less. Scroll bars shouldn’t be used frequently.
\item Uniformity in the format of screen/pages in your application/website.
\item Provision to search within your software application or website
\end{itemize}
\textbf{Accuracy}
\begin{itemize}
\item No outdated or incorrect data like contact information/address should be present.
\item No broken links should be present.
\item User Friendliness
\item Controls used should be self-explanatory and must not require training to operate
\item Help should be provided for the users to understand the application / website
\item Alignment with above goals helps in effective usability testing
\end{itemize}
\textbf{Tools required:} None(manual testing) \\
\textbf{Success criteria:} All of the following are successfully tested:
\begin{enumerate}
\item operations are not changed drastically from a traditional file sytem.
\item user can use the file system without any special tools.
\end{enumerate}
\end{enumerate}

\subsection{Entry and Exit Criteria:}

\begin{enumerate}
\item Test Plan
\begin{enumerate}[label=\Alph*]
\item \textbf{Test Plan Entry Criteria:} Code is complete and has been pushed to the Git repository.
\item \textbf{Test Plan Exit Criteria:} All functional requirments have been verified.
\item \textbf{Suspension and Resumption Criteria:} Testing will be suspended on critical
design flaws that will changes in redesign of critical components. Testing will resume when the coding is complete and code is reviewed successfully.
\end{enumerate}

\item Test Cycle
\begin{enumerate}[label=\Alph*]
\item \textbf{Test Cycle Entry Criteria:} When a module has been completed.
\item\textbf{ Test Cycle Exit Criteria:} All tests specified at the start of the testing have
completed successfully.
\end{enumerate}
\end{enumerate}

\subsection{Risks, Dependencies, Assumptions, Constraints}
\begin{table}[h]
\begin{tabular}{|p{2cm}|p{8cm}|p{4cm}|}
\hline
\textbf{Risk} & \textbf{Mitigation Strategy} & \textbf{Contingency} \\ \hline \hline
FUSE API changes & Use FUSE version numbers to run a static check while compiling for required version of FUSE. & Change operation code to new version. \\ \hline
External Library is no longer maintained & Try to use the latest version number of library available and keep a source ready for distribution. & Change to alternatice library. \\ \hline
Performance has degraded & Code with performance in mind, using fast algorithms and approaches. & Use profiling tools to detect memory issues and static code analysers for code checking. \\
\hline
\end{tabular}
\caption{Risk Management}
\label{tab: risk}
\end{table}

\subsection{Problem Reporting, Escalation, and Issue Resolution}
Each bug will be given a priority, which will determine when it is addressed in the current iteration. The bug priority may change due to other bugs, issues or re-evaluation of the bug by a peer review.

\section{Test Cases}
\subsection{Introduction}
The purpose of this Test Case document is to specify and communicate the specific
conditions which need to be validated to enable an assessment of the system. Test
Cases are motivated by many things but will usually include a subset of Use Cases,
performance characteristics and the risks the project is concerned with.
A separate test case document is prepared for each testing phase (unit,
integration, integrity, etc.) identified in the test plan. The test cases should be
organized into related groups that are meaningful to the project – i.e. test suites.

\subsection{File System Operations}
Testing the file system for implementations of required operations:
\begin{table}[h]
\begin{tabular}{|p{2cm}|p{2cm}|p{8cm}|}
\hline
\textbf{Operation} & \textbf{Status} & \textbf{Comment} \\ \hline
getattr & YES & checks whether the given path exits \\
readdir	 & YES & lists the contents of the given tag \\
access	& YES & checks for access to specified tag \\
truncate	 & YES & closes file after operation \\
destroy	 & YES & called on file system unmount \\
open		& YES & opens for file for access \\
release	& YES & releses file after access \\
mknod		& YES & creates new file \\
rename		& YES & renames files and folders \\
unlink		& YES & removes file from system \\
read		& YES & reads data from file \\
write		& YES & writes data to file \\
chmod		& YES & changes permissions \\
chown		& YES & changes owner\\
mkdir		& YES & creates new directory \\
rmdir		& YES & removes directory\\

symlink	 & INVALID & not required in KWEST \\ 
readlink & INVALID & not required in KWEST \\
link		& INVALID & not required in KWEST \\
utimens	& NO & not implemented \\
statfs	& NO & not implemented \\
fsync & INVALID & not required in KWEST \\

setxattr	& NO & not implemented \\
getxattr	& NO & not implemented \\
listxattr	& NO & not implemented \\
removexattr	& NO & not implemented \\
\hline
\end{tabular}
\caption{File system operations}
\label{fsop}
\end{table}

\subsection{Performance of file system operations}
Comparison of KWEST file system against underlying file system.
\textbf{Test Bench:}
\begin{itemize}
\item Operating System: Linux Mint 14 3.5.0-25-generic
\item Original file system: ext4 500GB disk with partition size 150GB
\item RAM: 4GB
\item swap: 8GB on disk
\item CPU utilization: average 4%
\end{itemize}
\textbf{Contents of Music folder imported into KWEST:}
\begin{itemize}
\item \textbf{Audio:} 17 files totalling 102.9MB
\item \textbf{Images:} 81 files totalling 196MB
\item \textbf{PDF:} 11 files totalling 17.9MB
\item \textbf{Video:} 4 files totalling 1GB
\item \textbf{Others:} 7 files totalling 7MB
\item \textbf{Total:} 120 files of size 1.3GB
\end{itemize}

\begin{table}[h]
\begin{tabular}{|p{3cm}|p{2cm}|p{7cm}|}
\hline
\textbf{Test} & \textbf{Time taken} & \textbf{Comment} \\ \hline
all files	&	120sec	& total file size imported was 1.3GB \\ \hline
videos	& 40sec	&	extracting metadata from videos is more expensiv compared to other file types \\ \hline
images	& 35sec & images having metadata take longer than those without \\ \hline
audio 	& 4sec	& audio files are the fastest to parse and load \\ \hline
PDF		& 2sec	& PDF files are parsed quickly as compared to other document types \\ \hline
forming associations & 2sec	 & time is proportional to number of common files in user tags \\
\hline
\end{tabular}
\caption{Performance tests for mounting KWEST}
\label{performancemount}
\end{table}

\textbf{Test Script:} The following is a simple test script used to evaluate file system operations. The scipt works by calculating the difference in time before and after the execution of operations.
\begin{lstlisting}[language=bash,frame=single]
#store current time
let DA=(`date +%s `)
#perform file system operation
ls -R kwest/src/mnt
#store new time
let DB=(`date +%s`)
#calculate the difference
let DC=$DB-$DA
#output the time taken
echo $DC
\end{lstlisting}

\begin{table}[h]
\begin{tabular}{|p{2cm}|p{1.5cm}|p{1.5cm}|p{7cm}|}
\hline
\textbf{Operation} & \textbf{ext4} & \textbf{KWEST} & \textbf{Comment} \\ \hline
list directory	&	500ms	&	550ms & there is no noticiable difference \\ \hline
read file	&	700ms & 850ms	& some extra time is taken to read a file depending on the amount of data being read. In general, there is no noticiable difference. \\ \hline
write file	&	1200ms & 2200ms	& (for 5MB text file) writing takes slightly more time, but the difference is within acceptable range. \\ \hline
read and write	&	1010ms & 2400ms & (for 5MB text file) reading and writing simultaenously does not produce any performance degradation. \\
\hline
\end{tabular}
\caption{Performance tests for using KWEST}
\label{performancetests}
\end{table}

\subsection{Profiling Code}
Code can be profiled using Manual methods, or using specific tools such as Valgrind, GDB, Splint etc. For testing KWEST, we have used the following profiling tools:
\subsection*{GDB}
GDB can be used to debug the file systemand check for memory leaks, errors and irregular operations. The sample output given below shows a clean mount and unmount of the KWEST file system.
\begin{lstlisting}[language=bash,frame=single]
$ gdb ./kwest
GNU gdb (GDB) 7.5-ubuntu
Copyright (C) 2012 Free Software Foundation, Inc.
License GPLv3+: GNU GPL version 3 or later 
<http://gnu.org/licenses/gpl.html>
This is free software: you are free to change and redistribute it.
There is NO WARRANTY, to the extent permitted by law.  
Type `show copying` and `show warranty` to see details.
This GDB was configured as "x86_64-linux-gnu".
For bug reporting instructions, please see:
<http://www.gnu.org/software/gdb/bugs/>...
Reading symbols from kwest/src/kwest....
(gdb) run -s -d -f mnt
Starting program: kwest/src/kwest -s -d -f mnt
[Thread debugging using libthread_db enabled]
Using host libthread_db library "/lib/x86_64-linux-gnu/libthread_db.so.1".
KWEST - A Semantically Tagged Virtual File System
...
...
[Inferior 1 (process 20863) exited normally]
(gdb) bt
No stack.
\end{lstlisting}

\begin{table}[h]
\begin{tabular}{|p{3cm}|p{2cm}|p{8cm}|}
\hline
\textbf{Test} & \textbf{Status} & \textbf{Possible Errors} \\ \hline
list directory	&	PASS &  I/O error, illegal operation, transport endpoint not connection, connection abort \\ \hline
read file & PASS & I/O error, illegal operation, access denied, database error \\ \hline
write file & PASS & I/O error, illegal operation, access denied, file system busy \\ \hline
mknod & PASS & Operation not permitted, I/O error, database error \\ \hline
unlink & PASS & Device busy, Operation not permitted, I/O error, database error \\ \hline
mkdir & PASS & Operation not permitted, I/O error, database error \\ \hline
rmdir & PASS & Device busy, Operation not permitted, I/O error, database error \\ \hline
chmod & PASS & Access denied, I/O error, device busy, database error \\ \hline
associations & PASS & memory error, segmentation fault, unconditional jump, I/O error \\ \hline
fuse main & PASS & incompatible version \\
\hline
\end{tabular}
\caption{GDB debugging for KWEST}
\label{tab:GDB}
\end{table}
