\chapter{User Manual}

KWEST is a virtual filesystem. This means that the folders and files represented by it are a part of its virtual organisation. Each file in KWEST represents an actual file stored somewhere on the underlying file system. The main focus of using KWEST is organisation.

\section{Installing}
\begin{itemize}
\item Once can get the latest copy of KWEST by downloading from the project hosting site: \url{https://code.google.com/p/kwest/downloads/list}
\end{itemize}

\section{Mounting and Unmounting}

\begin{itemize}
\item To mount a KWEST filesystem, the user needs to run the \textit{mount} script which will mount the file system in the specified folder. 
\item To unmount a KWEST filesystem, the user needs to run the \textit{fusermount -u} command with the argument \textit{path of KWEST mount point}. A successful unmount operation does not return any message.
\end{itemize}

\section{Importing Files and Folders}

\begin{itemize}
\item By default KWEST imports from the users \textbf{HOME} folder. It recursively scans all the subfolders and imports the folder hierarchy into KWEST. Hidden and System files are ignored by KWEST.

\item The user can also explicitly use the \textbf{import} tool to import files and folders into KWEST. The import tool accepts the folder to import as argument and imports all files and folders within it. It can work regardless of whether the file system is mounted or not.

\item While importing, the metadata of the file is used to organise each file. For eg: A music file contains the song's artist, which is used to categorise that file.
\end{itemize}

\section{Folder Structure}

The first layout when viewing a KWEST file system is called the Base folder. This folder contains a proper organisational view of the imported data, categorised by their metadata types. It contains the following folders:

\begin{enumerate}
\item \textbf{Files} \newline
This folder contains the user's folders as they  appear on their underlying file system. The folder is provided to give the user a quick access to their previous data organisation, as well as to demonstrate the superiority of a KWEST organisation.
\item \textbf{Audio} \newline
This folder contains all the files recognised by the system as being of type audio. Upon accessing the folder, each Audio file is further categorised by \textbf{Album}, \textbf{Artist}, \textbf{Genre}. If a particular metadata is absent for the audio file, KWEST categorises it in the Unkown folder. \newline
For e.g. If the audio file does not have any artist associated with it, it can be found in \textbf{UnknownArtist}.

\item \textbf{Image} \newline
This folder contains all the image files recognised by the system. Inside, the images are organised by ImageCreator and ImageDate. 
The ImageCreator subfolder is based on \textit{Creators} like software - Adobe Photoshop, or hardware - Camera Models. Each ImageCreator tag is further organised by ImageDate. 
The ImageDate folder contains images sorted by \textit{Month-Year} of creation. Eg: A picture taken on ``\textit{2nd March 1992}'' will appear under ``\textit{1992Mar}''.
As with Audio, files with metadata missing will appear under appropriate \textit{Unkown} subfolders.

\item \textbf{PDF} \newline
All PDF documents are tagged with the PDF tag. Each PDF document is organised by its \textit{Author,Publisher,Subject} and \textit{Title}. Files with metadata missing are appropriately tagged under \textit{Unknown} tags.

\item \textbf{Video} \newline
Currently, the system organises video based only on Length, with the categories being \textit{Short, Medium and Long}. A short video is anything with less than 1800s of playtime. Videos with play time equal to or greater than 5400s are considered Long, and anything between them is considered Medium.	
\end{enumerate}